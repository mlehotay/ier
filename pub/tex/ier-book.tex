% pub/tex/ier-book.tex
% Common rendering discipline for IER book-class PDFs.

% -----------------------------
% Column mode (publication-layer only)
%   Default: single column
%   Corpus book can opt into two-column body via:
%     -V header-includes="\\def\\iercolumnmode{two}"
%
% Implementation note:
%   Pandoc books are built with --toc. We keep the TOC in one column,
%   and switch to two columns immediately AFTER the TOC.
% -----------------------------
\usepackage{etoolbox}

% Default to single column unless overridden in header-includes.
\newcommand{\IERColumnMode}{one}
\ifdef{\iercolumnmode}{\renewcommand{\IERColumnMode}{\iercolumnmode}}{}

% Convenience switches (safe no-ops unless mode=two)
\newcommand{\IERMaybeTwoColumn}{%
  \ifdefstrequal{\IERColumnMode}{two}{\twocolumn}{}%
}
\newcommand{\IERMaybeOneColumn}{%
  \ifdefstrequal{\IERColumnMode}{two}{\onecolumn}{}%
}

% Keep TOC single-column; switch after TOC if mode=two.
\apptocmd{\tableofcontents}{\clearpage\IERMaybeTwoColumn}{}{}

% -----------------------------
% Paragraph / density discipline
% -----------------------------
\setlength{\parskip}{0pt}
\setlength{\parindent}{1em}

% -----------------------------
% Folios + running heads
%   - Page numbers required
%   - Chapter title only
%   - No section-level heads
% -----------------------------
\usepackage{fancyhdr}
\pagestyle{fancy}
\fancyhf{}

% Page number in outer header
\fancyhead[RO,LE]{\thepage}

% Chapter title only
\fancyhead[LO]{\nouppercase{\leftmark}}
\fancyhead[RE]{\nouppercase{\leftmark}}

% Prevent section titles from entering headers
\renewcommand{\sectionmark}[1]{}

\renewcommand{\headrulewidth}{0.4pt}
\renewcommand{\footrulewidth}{0pt}

% If we are in two-column mode and something forces one-column pages later,
% authors/scaffold can explicitly switch back using:
%   \IERMaybeOneColumn  ...  \IERMaybeTwoColumn
% (These are safe in single-column mode: they do nothing.)
